\section{Background}
The current system for monitoring rivers and bodies of water involves building small concrete huts that record and store data locally. The cost of building these huts makes them prohibitive and is inefficient with today's technology. A person is sent periodically to collect this information manually. Figure I shows one of these monitoring stations found along the River Beauly. 
\begin{figure}[H]
	\centering
	\includegraphics[width=0.7\linewidth]{images/geograph-2754945-by-Craig-Wallace.jpg}
	\caption{ SEPA Monitoring Station \citep{Wallace2012}}
	\label{fig:monStation}
\end{figure}
In 2005 the WFD reported 285 type 1a and 1b “at risk” bodies of water. \citep{SEPA2007} By 2006  253 stations such as the one pictured in Figure 1 were in place across Scotland representing 10\% of the country’s total water bodies and 26\% of the 989  “at risk” rivers as of 2009. \citep{SEPA2009}
\begin{figure}[H]
	\centering
	\includegraphics[width=0.7\linewidth]{images/monitoring_points.png}
	\caption{ Monitoring Stations in Scotland \citep{SEPA2006}}
	\label{fig:monStation}
\end{figure}
The reason for this can be attributed to the cost of these monitoring stations, with the solution proposed in this paper cost could be greatly reduced to a fraction of this. \citep{SEPA2016} Alternatives to this using more modern technology are still highly expensive and cost prohibitive \citep{TheIoTMarketplace2015}

\subsection{Open Source}

Many commercial applications developed are built with a propriety licence which prohibits the reading or sharing of the code. This is the antithesis of Open Source which has slowly been gaining traction from large companies, dispelling the myth that it's hobbyist code at best, as the Red Hat Inc. was valued at 35 bn. in it's recent acquisition \citep{Hammond2018} This is no better seen in the worldwide adoption of Apache, a http daemon that runs many websites from hobby to major company (Apple.com, Adobe.com etc.) \citep{W3techs2018}. Further project's such as the Linux kernel, originating from Linus Torvalds has seen a massive adoption, most strikingly in the supercomputing market where it has completely dominated the top 500 supercomputers (ranked by TFlops). \citep{Top5002018}
\\\\
The field of eco monitoring is a largely untapped field and with much of the current solutions being closed source and expensive \citep{TheIoTMarketplace2015} there is little interaction from the community and even less interoperability from the technologies deployed. There's also the consumer aspect that eco-conscious individuals want to help monitor the local environment. An attempt was made to crowd source the collection of atmospheric data \citep{OKLabStuttgart2015} across Europe and beyond with small sensor units developed during workshops.
\begin{figure}[H]
	\centering
	\includegraphics[width=0.7\linewidth]{images/openDataFeinstaubMap.jpg}
	\caption{ Air Quality Map \citep{OKLabStuttgart2018}}
	\label{fig:map}
\end{figure}
This model of crowd sourcing this deployment means anyone can get involved and ensures that the data is kept open to everyone. This is a key principle during this project as well and the Open Database License \citep{OpenDataCommons2011} will ensure the data can be used freely.