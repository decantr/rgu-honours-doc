\section{Background}
The current system for monitoring rivers and bodies of water involves building small concrete huts that record and store data locally. The cost of building these huts makes them prohibitive and is inefficient with today's technology. A person is sent periodically to collect this information manually. Figure I shows one of these monitoring stations found along the River Beauly. 
\\\\\\
\includegraphics[width=\linewidth]{images/geograph-2754945-by-Craig-Wallace.jpg}
\\\\\\
In 2005 the WFD reported 285 type 1a and 1b “at risk” bodies of water. \citep{SEPA2007} By 2006  253 stations such as the one pictured in Figure 1 were in place across Scotland representing 10\% of the country’s total water bodies and 26\% of the 989  “at risk” rivers as of 2009. (SEPA, 2009)⁠
\\\\\\
\includegraphics[width=\linewidth]{images/monitoring_points.png}
\\\\\\
The reason for this can be attributed to the cost of these monitoring stations, with the solution proposed in this paper cost could be greatly reduced to a fraction of this. 