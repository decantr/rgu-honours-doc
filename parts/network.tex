\section{Networks}

\begin{figure}[H]
	\hspace{-1cm}
	\centering
	\begin{tabular}{ccccc}
		\hline
		{} & ZigBee & Pi 3B+ WiFi & Pi 0 W & Bluetooth \\
		\hline
		IEEE & 802.15.04 & 802.11 b/g/n & 802.11 b/g/n & 802.15.1* \\
		Frequency (GHz) & 2.4 & 2.4/5 & 2.4 & 2.4 \\
		Main Application & Smart Devices & WLAN & WLAN & Low Bitrate Streams \\
		Bit Rate & 250kbps & 600 Mbps & 54 Mbps & 24 Mbps \\
		Range & 100m & 100m & 30m & 15m \\
		\hline 
	\end{tabular}
	\caption{ ZigBee vs. some other wireless network.\citep{RaspberryPiFoundation2018}, \citep{RaspberryPiFoundation2017} }
	\label{fig:networkcompare}
\end{figure}

The communication methods listen above all have there advantages and disadvantages. ZigBee has a significant advantage in this area as in addition to the above metrics it has a significantly lower battery usage than the other two protocols \citep{Leccese2014}. However this comes at a cost, both financially and to accessibility, as the ZigBee protocol requires an additional module to be attached to the device used to allow it to communicate. Further complicating things is these modules are an additional cost on top of the controller ( SoC ) and sensors required. \citep{zigbee2015} This additional complexity serves to increases the desirability of Wifi and Bluetooth as options. 

To connect these devices together, especially if Bluetooth is used, a method of daisy chaining them together is almost essential to keep costs to a minimum. This introduces many issues however as one of the main issues facing SEPA with it's current solution is vadalism \citep{SEPA2018} and these sensors could be an attractive target. The removal of a device near the start of a chain either through a fault, theft etc. could affect many more sensors further along having a cascading effect. This is not ideal and makes Bluetooth a very unattractive option.

