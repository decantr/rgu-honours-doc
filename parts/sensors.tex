\section{Sensors}
\subsection{Water Level}
The collection of the river data will be the trickiest part of the project. To collect water level currently a wire is laid across the river and a buoy attached to the middle, the sag in the wire can then be used to calculate depth. \citep{SEPA2016} An alternative solution involves using an ultrasonic sensor to judge the distance between the sensor and the water. \citep{AravindJayan2016} This method has the advantage of being a low physical footprint solution to monitoring. 

\begin{figure}[H]
	\centering
	\includegraphics[width=0.7\linewidth]{images/HC-SR04_SPL.jpg}\\
	\caption{ HC SR04 Ultrasonic Sensor \citep{MouserElectronics}}
	\label{fig:hcsr04}
\end{figure}

\subsection{pH Sensor}
The sensing of the pH level is somewhat more difficult and requires a specialist tool. The EZO-pH Embedded pH Circuit \citep{AtlasScientific2018a} which allows a digital signal to be read through i2c or UART by a connected device. \cite{Dey2018}

\subsection{Temperature Sensor}

Types of sensors that could be used. Elements that need to be detected. 