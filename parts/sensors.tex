\section{Sensors}
\subsection{Water Level}
The collection of the river data will be the trickiest part of the project. Currently, to collect water level data, a wire is laid across the river and a buoy attached to the middle. The sag in the wire can then be used to calculate depth \citep{SEPA2016}. An alternative solution involves using an ultrasonic sensor to judge the distance between the sensor and the water \citep{AravindJayan2016}. An example of this sensor is shown in Figure \ref{fig:hcsr04}. This method has the advantage of being a low physical footprint solution to monitoring. 

\begin{figure}[H]
	\centering
	\includegraphics[width=0.7\linewidth]{images/HC-SR04_SPL.jpg}\\
	\caption{ HC SR04 Ultrasonic Sensor \citep{MouserElectronics}}
	\label{fig:hcsr04}
\end{figure}

\subsection{pH Sensor}
The sensing of the pH level is somewhat more difficult and requires a specialist tool. The EZO-pH Embedded pH Circuit \citep{AtlasScientific2018a} which allows a digital signal to be read through i2c or UART by a connected device. \cite{Dey2018} One issue with this sensor is that is an analogue device and thus if a Raspberry Pi is used we will require an ADC (Analogue to Digital Converter) to hook it up, if however a Arduino is used this will not be an issue.

\subsection{Temperature Sensor}
To collect the temperature a wide range of sensors can be used with the only requirement being they are waterproof. The most accessible of these is the DS18B20 \citep{DS18B20} which communicates over the 1-Wire protocol. This is particularly useful to us as it is a digital sensor meaning we will not need the additional ADC if a Raspberry Pi (Zero) is used. Once connected this can be polled for the current temperature at the sensor. \citep{Monk2013}